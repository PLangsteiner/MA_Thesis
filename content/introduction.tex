\chapter{Introduction}
\label{chap:introduction}
\paragraph{Motivation}
\begin{itemize}
    \item Find an interesting object/scene for a game or a movie (e.g., Shopping Mall, Office, etc.)
    \item Digitizing the object/scene allows for manipulating it as the user sees fit
\end{itemize}

\paragraph{Problem Description}
\begin{itemize}
    \item Current methods allow for accurate reconstruction of geometry and appearance under lighting conditions of the original scene
    \item Fall short in relighting scenarios due to the lack of precise, spatially varying material parameters
    \item To fully capture the object/scene, we need to have precise material parameters
    \item Current 3D reconstruction methods allow for either accurate parameters for a single object or often inaccurate estimation of parameters for a scene
    \item Biggest challenge in current methods are baked-in lighting conditions in material parameters
\end{itemize}

\paragraph{Contributions}
\begin{itemize}
    \item Advances in 2D material prediction allow for accurate prediction of material parameters
    \item This thesis focuses on fusing 2D material data into 3D geometry using a combination of learning-based and projection-based approaches.
    \item Learning-based approach learns the material parameters using image-based losses
    \item Using a simple L1 loss, the predictions are compared to the current renders of the material
    \item Projection-based approach projects the material parameters onto the 3D geometry using Gaussian ray tracing
    \item Using ray-tracing from each view, the intersected Gaussians get assigned the material parameter directly
    \item This allows for a much faster approach of applying the material parameters to each Gaussian
    \item To enhance fine-scale accuracy, we further introduce a neural refinement step using a multilayer perceptron (MLP), which takes ray-traced material features as input and produces detailed adjustments.
    \item This allows for a much more accurate approach of applying the material parameters to each Gaussian
\end{itemize}

\paragraph{Outline}
\begin{itemize}
    \item Chapter 2: Background
    \item Chapter 3: Related Work
    \item Chapter 4: Method
    \item Chapter 5: Results
    \item Chapter 6: Conclusion
\end{itemize}



