\chapter{Background}
\label{chap:background}

\section{Physically Based Rendering}
- Rendering Equation
- Path Tracing
- Real-Time Rendering
- Scene needs Geometry, materials, lights, and cameras

\subsection{Scene Components}

\paragraph{Materials}
- Materials that are needed are Base Color, Roughness, Metallic

\paragraph{Geometry}
- Geometry is normally defined by a triangle mesh
- Other representations are point clouds, implicit surfaces, and volumetric data
- Gaussians can represent geometry by simulating a point cloud

\paragraph{Lighting}
- Point lights
- Environment lights

\paragraph{Cameras}
- Pinhole Camera Model for synthetic data
- Real-world cameras are more complex
- Need to know the camera intrinsics and extrinsics for correct rendering

\subsection{Bidirectional Reflectance Distribution Function}
- BRDF is a function that describes the reflectance of a surface
- Multiple BRDF models exist
- Cook-Torrance BRDF is a popular model for physically based rendering in real-time

\section{3D Reconstruction}

- Trying to reconstruct a scene from multiple images from different views
- Synthetic tasks are often a disconnected series of images
- Real-world tasks are often videos, but are harder to reconstruct because of:
    - Motion blur
    - Occlusion
    - Varying lighting
    - Varying camera parameters
    - Varying scene geometry
    - Varying scene materials
    - Varying scene lighting

- possibly history of 3D Reconstruction

\subsection{Structure from Motion}
- Structure from Motion is a way to create a 3D Point cloud by comparing images for reference points that they have in common
- Most important Structure from Motion is Colmap

\subsection{NeRF}
- NeRF was one of the first Neural 3D Reconstruction methods
- NeRF works by training a neural network to represent a 3D scene
- It uses Volume Rendering to render the scene
- NeRF pioneered neural scene representation and differentiable rendering, which Gaussian Splatting builds upon.

\subsection{Gaussian Splatting}
- Gaussian Splatting gave a huge boost to 3D Reconstruction because it was able to render the scene at a much higher frame rate than NeRF
- Gaussian Splatting places 3D Gaussians in a scene and uses rasterization to render these Gaussians
- Using an image based loss, the Gaussians are then optimized to fit the original images
- With enough images this results in a high fidelity appearance reconstruction of the scene
- Gaussian Splatting biggest advantage is that it is an explicit representation of the scene

\subsection{Gaussian Ray Tracing}
- Gaussian Ray Tracing uses Ray Tracing to render Gaussians
- There are two main approaches to Gaussian Ray Tracing:
    - Using volume rendering to render the Gaussians
    - Using the point of most response to approximate the contribution of each Gaussian
- The ray tracing approach allows for direct view sorting and projection of attributes to the Gaussians

\section{Multilayer Perceptron}
- A Multilayer Perceptron (MLP) is a type of artificial neural network that consists of multiple layers of neurons
- Each neuron receives input from the previous layer and applies a weighted sum of the inputs to a non-linear activation function
- The output of the last layer is the final prediction
- MLPs are used in a variety of applications, including image classification, object detection, and natural language processing

\paragraph{Softmax}
- The Softmax function is a non-linear activation function that is used in the output layer of a Multilayer Perceptron
- It gives a probability distribution for each output neuron
- This can be used as a weighting function

\section{Diffusion}
- Diffusion models are generative models that learn to reverse a gradual noising process
- They work by training a neural network to predict and remove noise from images step by step
- The forward process gradually adds Gaussian noise to data over multiple timesteps
- The reverse process learns to denoise by predicting the noise at each timestep
- Diffusion models can generate high-quality images by starting from pure noise
- They can be conditioned on text, images, or other modalities for controlled generation
