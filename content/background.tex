\chapter{Background}
\label{chap:background}

\section{Physically Based Rendering}
- Rendering Equation
- Path Tracing
- Real-Time Rendering
- Scene needs Geometry, materials, lights, and cameras

\subsection{Scene Components}

\paragraph{Materials}
- Materials that are needed are Base Color, Roughness, Metallic

\paragraph{Geometry}
- Geometry is normally defined by a triangle mesh
- Other representations are point clouds, implicit surfaces, and volumetric data
- Gaussians can represent geometry by simulating a point cloud

\paragraph{Lighting}
- Point lights
- Environment lights

\paragraph{Cameras}
- Pinhole Camera Model for synthetic data
- Real-world cameras are more complex
- Need to know the camera intrinsics and extrinsics for correct rendering

\subsection{Bidirectional Reflectance Distribution Function}
- BRDF is a function that describes the reflectance of a surface
- Multiple BRDF models exist
- Cook-Torrance BRDF is a popular model for physically based rendering in real-time

\section{3D Reconstruction}

- Trying to reconstruct a scene from multiple images from different views
- Synthetic tasks are often a disconnected series of images
- Real-world tasks are often videos, but are harder to reconstruct because of:
    - Motion blur
    - Occlusion
    - Varying lighting
    - Varying camera parameters
    - Varying scene geometry
    - Varying scene materials
    - Varying scene lighting


\subsection{Structure from Motion}

\subsection{NeRF}

\subsection{Gaussian Splatting}

\subsection{Gaussian Ray Tracing}

\section{Diffusion}